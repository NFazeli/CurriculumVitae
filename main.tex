\documentclass[11pt,a4paper,sans]{moderncv}

\moderncvstyle{classic}                             % style options are 'casual' (default), 'classic', 'banking', 'oldstyle' and 'fancy'
\moderncvcolor{blue}                               % color options 'black', 'blue' (default), 'burgundy', 'green', 'grey', 'orange', 'purple' and 'red'
\usepackage[utf8]{inputenc}                       % if you are not using xelatex ou lualatex, replace by the encoding you are using

\usepackage[scale=0.78]{geometry}

\name{Nastaran}{Fazeli}
\address{Köln}{Germany}
\extrainfo{Github: NFazeli}  


\begin{document}
\makecvtitle

With my diverse background in Science and Engineering I have strong expertise in statistics and programming. 
As a curious researcher I enjoy to challenge myself with new topics and employ my problem-solving skills to bring the given task to the next level.
After several years in fundamental research, I am now looking forward to move to a faster paced environment. 
I have a learner's spirit, feeling enthusiastic about the opportunities to become an expert in agile software development and AI.

\section{Technical Skills}
\cvlistitem{7+ years experience in \textsc{Python} programming on a daily basis, including libraries such as \textsc{NumPy}, \textsc{SciPy}, \textsc{Pandas}, \textsc{Matplotlib}, \textsc{Scikit-learn}, \textsc{NLTK}, \textsc{AstroPy}: developing pipelines, improving old pipelines and bug-fixing of scripts. Also Sporadic use of SQL and NoSQL in several projects}

\cvlistitem{Good knowledge of applied statistics and machine learning, including application of MCMC or clustering algorithms in recent research}

\cvlistitem{10+ years experience with \textsc{Linux} and \textsc{Windows} operating systems, as well as with \textsc{Latex} and \textsc{MS Office}}

\cvlistitem{First experience using \textsc{Tensorflow}, \textsc{Keras}, \textsc{Docker}, \textsc{OpenCV} and \textsc{Tesseract} through on-going hobby project "live detection and solving of Sudoku puzzles"}

\cvlistitem{Interested in big data analysis, first experience with \textsc{Apache Spark} and \textsc{Databricks} environment}

\section{Work Experience}
\cventry{since 04/2016}{Research Assistant}{I. Physikalisches Institut, University of Cologne}{Germany}{}{My research was part of an international collaboration studying the mechanisms that lead to feeding and feedback of supermassive black holes especially in the nearby universe.
During my research I gained experience on data reduction, analysis and visualization working with large observational data sets from the world's foremost astronomical observatories. For this I developed my own Python scripts and modified existing pipelines. I presented the results of my research at many international conferences and published papers in leading astronomical journals.}
\cventry{2014-2016}{Student assistant in admissions office}{Bonn-Cologne Graduate School of Physics and Astronomy}{}{}{For this job I prepared applicants profiles for the selecting Committee under strict deadlines.
Created and maintained database for the admission office, using Microsoft Excel and Access.
Organized events during the admission academy week and communicated with students and staff.
Consulted students about rules, regulations and opportunities at University of Cologne. 
Provided administrative support services to students during semester start-ups.}

\newpage
\section{Education}
\cventry{2016--2020}{PhD in Experimental Physics (Dr.~rer.~nat.)}{University of Cologne}{Germany}{\textit{1.0 (magna cum laude)}}{Star formation and gas flows in the central kiloparsec of nearby galaxies observed by SINFONI} 
\cventry{2013--2016}{Master of Science in Physics (M.Sc.)}{University of Cologne}{Germany}{\textit{1.1}}{Star formation and AGN activity in the nearby Seyfert galaxy NGC 7496}
\cventry{2009--2013}{Bachelor of Science in Physics (B.Sc.)}{University of Leipzig}{Germany}{\textit{1.9}}{Optical/UV spectra of quasar accretion disks with gaps and holes}
\cventry{2005--2009}{Bachelor of Science in Industrial Engineering (B.Sc.)}{Islamic Azad University, Tehran North Branch}{Iran}{\textit{GPA -- 15.3/20}}{}


\section{Transferable Skills}
\subsection{Communication}
\cvlistitem{Publication of research results in peer-reviewed journal, 3 as first author and 7 as co-author}
\cvlistitem{Presentation of results in talks on international conferences and workshops, e.g. in Santiago de Chile, Liverpool, Prague, Zermatt, Chania}
\cvlistitem{Participated at \href{http://www.theport.ch}{\underline{THE Port humanitarian hackathon}} 2016 hosted by CERN}
\cvlistitem{Team work in international collaborations and a multi-cultural working group}

\subsection{Leadership and organization}
\cvlistitem{Supervision of 2 bachelor/master students. Teaching of $> 50$ students in lab courses and master of astrophysics tutorials.}

\cvlistitem{Organized events in the admission academy week of Bonn-Cologne graduate school}
\cvlistitem{Representation of (doctoral) students in SFB 956 student council and organization of common activities, including yearly retreat}
\cvlistitem{Proposal writing for grant and observation time proposals and 10 nights observing run at a leading astronomical observatory}

\subsection{Motivation and personal drive}
\cvlistitem{Have a learner's spirit and good at solo learning and in a team, especially interested in new technologies}
\cvlistitem{Always curious and in research to reach a deeper understanding of problems in my surroundings}

\section{Languages}
\cvlistitem{English: Fluent}%{Comment}
\cvlistitem{German: Advanced (currently doing a C1 intensive course)}
\cvlistitem{Persian: Mother-tongue}


\section{Relevant Certificates}
\cvlistitem{Coursera applied Data Science with Python Specialization \\ \href{https://www.coursera.org/account/accomplishments/certificate/YE5LUCQL8WCJ} {\underline{"Introduction to Data Science in Python"}}, \href{https://coursera.org/share/7f2d21cb098c1e24007dc857ffec48a0}{\underline{"Applied Machine Learning in Python"}}, \href{https://coursera.org/share/243cbcfb0a6566f006d36456fa34aaf5}{\underline{"Applied Text Mining in Python"}}
\href{https://www.coursera.org/account/accomplishments/certificate/ZSHQRPZ5ZT8R}{\underline{"SQL for Data Science"}}}

\end{document}
